% ====== TAREA 1 MATEMATICAS APLICADAS ======

\documentclass{article}
\usepackage[utf8x]{inputenc}
\usepackage{amsmath}
\usepackage{amsfonts}
\usepackage{amssymb}
\usepackage{graphicx}
\usepackage{enumitem}
\usepackage[text={20cm,25cm},centering,top=1.5cm,bottom=1.5cm,letterpaper,showframe=false]{geometry}
\renewcommand{\baselinestretch}{1.5}
\parindent  = 0mm
\parskip    = 4mm

\begin{document}
\title{Tarea 1}
        \author{Careaga Carrillo Juan Manuel \\ Quiróz Castañeda Edgar \\ Soto Corderi Sandra del Mar}
        \date{Lunes 21 de Septiembre del 2018}
        \maketitle

	\begin{enumerate}
   	% Ejercicio 1
   	\item {
   Dibujar la región W definida por las superficies $x + 2y + 3z = 6$ y $z = 0$. Expresar la integral triple $\iiint_Wf(x,y,z)dV$ de las seis formas posibles como integrales iteradas.\\

	}

	% Ejercicio 2
   \item {
    Dibujar la región W descrita por la integral iterada
    \begin{center}
    $\int_{0}^{1}\int_{z^3}^{\sqrt{z}}\int_{0}^{4-x}dydxdz$. 
    \end{center}
    Calcular su volumen.\\
		     
	}


	% Ejercicio 3
   \item {
    Integrar la función $f(x,y) = x^2 + 2xy^2 + 2$ sobre la región $D$ acotada por la gráfica de $y = -x^2 + x$, el eje $x$, y las rectas $x = 0$ y $x = 2$.\\


	}

	% Ejercicio 4
   \item {
   Usar integrales dobles para calcular el área de una elipse con semiejes $a$ y $b$.\\

	}

	% Ejercicio 5
   \item {
   Mostrar que al evaluar $\iint_DdA$, donde $D$ es una región $y$-simple, se reproduce la fórmula del cálculo de una variable para el área entre dos curvas.\\

	

    }

	% Ejercicio 6
   \item {
   Describir la región que se encuentra entre el cono $z = \sqrt{x^2 + y^2}$ como una región elemental.\\

	
    }
    
    % Ejercicio 7
   \item {
   Hallar el volumen acotado por el paraboloide $z = 2x^2 + y^2$ y el cilindro $z = 4 - y^2$\\

	
    }
    
    % Ejercicio 8
   \item {
    \begin{enumerate}
	\item
	Investigar qué es el centro de masa de un cuerpo y cómo se puede calcular usando integrales múltiples.\\	
	
	
	\item    
	La densidad en un punto $P$ de un cubo sólido de lado $a$ es directamente proporcional al cuadrado de la distancia del punto $P$ a una esquina fija del cubo. Encontrar el centro de masa.\\    
    
    \end{enumerate}
    }
    
\end{enumerate}
\end{document}
